
\LaTeX 에서 다양한 표를 제작하고 멋지게 표현하는데 다양한 팩키지가 존재한다. 
경우에 따라서 페이지를 넘어가는 문서를 제작해야하는 경우 \verb*|threeparttable| 이 유용하다.

\begin{verbatim}
\usepackage{booktabs}
\usepackage{multirow}
\usepackage{threeparttable}
\usepackage{caption}
\usepackage{subcaption}
\end{verbatim}


\subsection{간단한 표}

\begin{center}
	\begin{tabular}{ c c c }
		cell1 & cell2 & cell3 \\ 
		cell4 & cell5 & cell6 \\  
		cell7 & cell8 & cell9    
	\end{tabular}
\end{center}

\subsection{윤곽을 갖는 표}

\begin{center}
	\begin{tabular}{ |c|c|c| } 
		\hline
		cell1 & cell2 & cell3 \\ 
		cell4 & cell5 & cell6 \\ 
		cell7 & cell8 & cell9 \\ 
		\hline
	\end{tabular}
\end{center}

\subsection{TD, TR 구분된 표}

\begin{center}
	\begin{tabular}{||c c c c||} 
		\hline
		Col1 & Col2 & Col2 & Col3 \\ [0.5ex] 
		\hline\hline
		1 & 6 & 87837 & 787 \\ 
		\hline
		2 & 7 & 78 & 5415 \\
		\hline
		3 & 545 & 778 & 7507 \\
		\hline
		4 & 545 & 18744 & 7560 \\
		\hline
		5 & 88 & 788 & 6344 \\ [1ex] 
		\hline
	\end{tabular}
\end{center}

\subsection{표 제목과 캡션}

표 \ref{table:1} 은 \LaTeX 에서 참조하여 추적을 보여주는 한 사례다.

\begin{table}[h!]
	\centering
	\begin{tabular}{||c c c c||} 
		\hline
		Col1 & Col2 & Col2 & Col3 \\ [0.5ex] 
		\hline\hline
		1 & 6 & 87837 & 787 \\ 
		2 & 7 & 78 & 5415 \\
		3 & 545 & 778 & 7507 \\
		4 & 545 & 18744 & 7560 \\
		5 & 88 & 788 & 6344 \\ [1ex] 
		\hline
	\end{tabular}
	\caption{Table to test captions and labels}
	\label{table:1}
\end{table}

\subsection{고정폭 테이블}

\blindtext

\begin{table}[h!]
	\centering
	\begin{tabular}{ | m{5em} | m{1cm}| m{1cm} | } 
		\hline
		cell1 dummy text dummy text dummy text& cell2 & cell3 \\ 
		\hline
		cell1 dummy text dummy text dummy text & cell5 & cell6 \\ 
		\hline
		cell7 & cell8 & cell9 \\ 
		\hline
	\end{tabular}
	\caption{Fixed length 고정폭 표}
    \label{table:2}
\end{table}

\subsection{표선 굵기 조절 테이블}

\begin{table}[h!]
	\centering
	\begin{tabular}{ |p{3cm}|p{3cm}|p{3cm}|  }
		\hline
		\multicolumn{3}{|c|}{Country List} \\
		\hline
		Country Name     or Area Name& ISO ALPHA 2 Code &ISO ALPHA 3 \\
		\hline
		Afghanistan & AF &AFG \\
		Aland Islands & AX   & ALA \\
		Albania &AL & ALB \\
		Algeria    &DZ & DZA \\
		American Samoa & AS & ASM \\
		Andorra & AD & AND   \\
		Angola & AO & AGO \\
		\hline
	\end{tabular}
	\caption{Line Length 선굵기 변경 표}
	\label{table:3}
\end{table}

