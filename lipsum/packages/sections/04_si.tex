
국제단위계(Système international d’unités, 약칭 SI)는 도량형의 하나로, MKS 단위계(Mètre-Kilogramme-Seconde)이라고도 불린다. 국제단위계에서는 7개의 기본 단위가 정해져 있다. 이것을 SI 기본 단위(국제단위계 기본 단위)라고 한다.

\begin{table}[h]
	\begin{tabular}{@{}lll@{}}
		\toprule
		물리량                       & 이름                        & 기호                       \\ \midrule
		\multicolumn{1}{|l|}{길이}  & \multicolumn{1}{l|}{미터}   & \multicolumn{1}{l|}{m}   \\ \midrule
		\multicolumn{1}{|l|}{질량}  & \multicolumn{1}{l|}{킬로그램} & \multicolumn{1}{l|}{kg}  \\ \midrule
		\multicolumn{1}{|l|}{시간}  & \multicolumn{1}{l|}{초}    & \multicolumn{1}{l|}{s}   \\ \midrule
		\multicolumn{1}{|l|}{전류}  & \multicolumn{1}{l|}{암페어}  & \multicolumn{1}{l|}{A}   \\ \midrule
		\multicolumn{1}{|l|}{온도}  & \multicolumn{1}{l|}{켈빈}   & \multicolumn{1}{l|}{K}   \\ \midrule
		\multicolumn{1}{|l|}{물질량} & \multicolumn{1}{l|}{몰}    & \multicolumn{1}{l|}{mol} \\ \midrule
		\multicolumn{1}{|l|}{광도} & \multicolumn{1}{l|}{칸델라}    & \multicolumn{1}{l|}{cd} \\ \bottomrule
	\end{tabular}
\end{table}

\subsection{숫자 간격}

\begin{itemize}
	\item 만단위 숫자 \verb*|\num{12890}| : \num{12890}
	\item 천단위 숫자 \verb*|\num{1289}|: \num{1289}
	\item 소수점 숫자 \verb*|\num{.346}|: \num{.346}
	\item 소수점 숫자 \verb*|\num{1.23e-6}|: \num{1.23e-6}
\end{itemize}


\subsection{각도}

\begin{itemize}
	\item 도형의 각도 \verb*|\ang{45}| : \ang{45}	
	\item 지도 위경도 \verb*|\ang{60;2;3}|: \ang{60;2;3}
\end{itemize}


\subsection{길이와 면적}

\begin{itemize}
	\item 거리 \verb*|\si{\kilo\metre}| : \si{\kilo\metre}	
	\item 면적 \verb*|\si{\kilo\metre\squared}|: \si{\kilo\metre\squared}
\end{itemize}

\subsection{가속도}

\begin{itemize}
	\item 가속도 \verb*|\si{\metre\per\square\second}|: \si{\metre\per\square\second}	
	\item 중력 가속도 \verb*|\SI{9.78}{\metre\per\square\second}|: \SI{9.78}{\metre\per\square\second}
\end{itemize}
